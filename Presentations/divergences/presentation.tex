\documentclass{beamer}
\usepackage[utf8]{inputenc} % allow utf-8 input
\usepackage[T1]{fontenc}    % use 8-bit T1 fonts
\usepackage{hyperref}       % hyperlinks
\usepackage{url}            % simple URL typesetting
\usepackage{booktabs}       % professional-quality tables
\usepackage{amsfonts}       % blackboard math symbols
\usepackage{nicefrac}       % compact symbols for 1/2, etc.
\usepackage{microtype}      % microtypography
\usepackage{amsmath}
\DeclareMathOperator*{\argmax}{argmax}
\DeclareMathOperator*{\argmin}{argmin}
\usepackage{tikz}
\usetikzlibrary{calc}
\usetikzlibrary{bayesnet}
\usetikzlibrary{arrows}
\usepackage{color}
\usepackage{array}
\usepackage{dsfont}
\usepackage{multirow, graphicx}
 \usepackage{float}
\newcolumntype{C}[1]{>{\centering\arraybackslash}p{#1}}
\newcolumntype{R}[1]{>{\raggedleft\arraybackslash}p{#1}}
\newcolumntype{L}[1]{>{\raggedright\arraybackslash}p{#1}}
\usepackage{caption}
\usepackage{subcaption}
\usepackage{pifont}
\usepackage{xcolor}
\newcommand{\cmark}{\textcolor{green!80!black}{\ding{51}}}
\newcommand{\xmark}{\textcolor{red}{\ding{55}}}
\urlstyle{same}
% \usetheme{Boadilla}

\title{Introduction to Differentiable Probabilistic Models}
% \subtitle{Using Beamer}
\author{Bill Watson}
% \institute{University of ShareLaTeX}
\date{\today}

\begin{document}

\begin{frame}
\titlepage
\end{frame}


\begin{frame}
\frametitle{Primer: Standard Machine Learning}
\begin{itemize}
\item Usually, we are given a set $\mathcal{D} = \{ X, y\}$
  \begin{equation*}
  X =
    \begin{bmatrix}
        x_{11} & x_{12} & \dots  & x_{1n} \\
        x_{21} & x_{22} & \dots  & x_{2n} \\
        \vdots & \vdots & \ddots & \vdots \\
        x_{m1} & x_{m2} & \dots  & x_{mn}
    \end{bmatrix}
    \quad \quad
    y =
    \begin{bmatrix}
      y_1 \\
      y_2 \\
      \vdots \\
      y_n
    \end{bmatrix}
  \end{equation*}
  where $X$ is our data matrix, and $y$ are our labels.
\pause
\item Attempt to fit a model $f$ parameterized by $\theta$ with respect to
an object function $\mathcal{L}$
\begin{equation*}
  \text{minimize} \; \mathcal{L} \big( f( X;\theta), \; y \big)
\end{equation*}
\end{itemize}
\end{frame}


\begin{frame}
\frametitle{Reframed as a Probabilistic Model}
\begin{itemize}
\item Consider our model is a probability distribution $Q$
\begin{itemize}
  \item No longer have labels $y$
  \item But have probabilties and sampling
\end{itemize}
\pause
\item Consider that our data is sampled from the real world $P$:
\begin{equation*}
  \begin{gathered}
  X \sim P \\
  \text{minimize} \; \mathcal{L} \big( f(X;\theta) \big) \\
\end{gathered}
\end{equation*}
\pause
\item Examples:
\begin{itemize}
  \item Classification: Fitting two multinomial distributions
  \item Regression: Fitting a Normal centered around the line of best fit
\end{itemize}
\end{itemize}
\end{frame}

\begin{frame}
  \frametitle{How do we "fit" Distributions?}
  \begin{itemize}
    \item Fitting two distirbutions implies minimizing their difference,
      i.e. "distance"
    \item This "distance" is known as the divergence between the true distribution
    $P$ and the approximate "learned" distribution Q.
    \pause
    \item Divergences must statisfy 2 properties:
    \begin{itemize}
      \item $D(P \parallel Q ) \geq 0 \quad \forall P, Q \in S$
      \item $D(p \parallel Q ) = 0 \iff P = Q$
    \end{itemize}
  \end{itemize}

\end{frame}

\begin{frame}
  \frametitle{The Kullback-Leibler Divergence}
  \begin{itemize}
    \item The KL Divergence for distributions $P$ and $Q$ is defined as:
    \begin{equation*}
      D_{KL} (P \parallel Q) = \int_{-\infty}^{\infty} p(x)\log \left({\frac {p(x)}{q(x)}}\right)\,dx
    \end{equation*}
    \pause
    \item Note: this implies that the KL Divergence is NOT symmetric:
    \begin{equation*}
      D_{KL} (P \parallel Q) \not= D_{KL} (Q \parallel P)
    \end{equation*}
    \item But we will come back to this later...
  \end{itemize}
\end{frame}


\begin{frame}
  \frametitle{Directionality: Forward KL}
  \begin{itemize}
    \item This direction is known as the Forward KL
    \begin{equation*}
      D_{KL} (P \parallel Q) = \int_{-\infty}^{\infty} p(x)\log \left({\frac {p(x)}{q(x)}}\right)\,dx
    \end{equation*}
    \pause
    \item The Forward KL will sample from $P$, and evaluate the log probabilties of $P$ and $Q$
    \pause
    \item Note: Maximum Likelihood Estimation is equivalent to minimizing the Forward KL
  \end{itemize}
\end{frame}


\begin{frame}
  \frametitle{Digression: KL to Cross-Entropy}
  \begin{itemize}
    \item The KL Divergence can be decomposed into familiar terms:
    \begin{equation*}
      \begin{aligned}
        D_{KL} (P \parallel Q) =& -\sum_{x \in \mathcal {X}} p(x)\log \left({\frac {q(x)}{p(x)}}\right) \\
        \pause
        =& - \sum_{x \in \mathcal {X}} p(x)\log q(x) + \sum_{x \in \mathcal {X}} p(x)\log p(x) \\
        \pause
        =& \underbrace{H(P, Q)}_{Cross-Entropy} - \underbrace{H(P)}_{Entropy} \\
        \pause
        \propto& \underbrace{H(P, Q)}_{Cross-Entropy}
      \end{aligned}
    \end{equation*}
    \pause
    \item If we consider $P(x_i) = y_i = \{0, 1\}$ and $Q(x_i) = \sigma(f(x_i))$:
    \begin{equation*}
      D_{KL} (P \parallel Q) \propto - \Big[ y_i \log \sigma\big(f(x_i)\big) + (1 - y_i) \log (1 - \sigma \big(f(x_i))\big) \Big]
    \end{equation*}
  \end{itemize}
\end{frame}


\begin{frame}
  \frametitle{Forward KL: Learning a Normal Distribution}
  \begin{itemize}
    \item Lets learn a simple Normal Distribution
    \item SHOW PLOT
  \end{itemize}
\end{frame}

\begin{frame}
  \frametitle{Forward KL: Learning a Normal Distribution}
  \begin{itemize}
    \item SHOW STATS
    \item SHOW PLOT
    \item However, this is boring...
  \end{itemize}
\end{frame}


\begin{frame}
  \frametitle{Digression: Mixture Models}
  \begin{itemize}
    \item We can build a $K$ multi-modal distribution, with weights
      $\mathbf{\pi}$, as follows:
    \begin{equation*}
      \begin{aligned}
      z \; \sim& \; \text{Categorical}(\mathbf{\pi}) \\
      x \,|\, z = k \; \sim& \; \text{Normal}(\mu_k, \sigma_k) \\
    \end{aligned}
    \end{equation*}
    \item We can calculate log probabilities by marginalizing out $z$:
    \begin{equation*}
      \log p(x) = \log \sum_{k=1}^{K} \underbrace{p(z = k)}_{\text{Categorical}} \cdot \underbrace{p(x \,|\, z=k)}_{\text{Normal}}
    \end{equation*}
  \end{itemize}

\end{frame}


\begin{frame}
  \frametitle{Forward KL: Fitting a Normal to a Bimodal Distribution}
  \begin{itemize}
    \item SHOW Models
    \item SHOW TRAINING + RET
  \end{itemize}
\end{frame}


\begin{frame}
  \frametitle{Forward KL: Loss Landscape}
  \begin{itemize}
    \item Show Loss landscape
    \item SHOW ret
    \item Show eq + reason
  \end{itemize}
\end{frame}

\begin{frame}
  \frametitle{Forward KL: Zero-Avoiding}
  \begin{equation*}
    D_{KL} (P \parallel Q) = \int_{-\infty}^{\infty} \overbrace{p(x)}^{\text{Constant}}\log \left({\frac {\overbrace{p(x)}^{\text{Constant}}}{\underbrace{q(x)}_{Variable}}}\right)\,dx
  \end{equation*}
  \begin{itemize}
    \item $p(x)$ is constant-valued, $q(x)$ is variable
    \item If $Q$ does not support $P$, then we will sample a point that has a low
      probability with respect to $Q$
    \pause
    \item As $q(x) \to 0$, our loss $D_{KL} \to infty$
    \pause
    \item Hence, the optimal solution is for $Q$ to cover $P$, i.e. averaging
  \end{itemize}
\end{frame}


\begin{frame}
  \frametitle{Directionality: Reverse KL}
  \begin{equation*}
    D_{KL} (P \parallel Q) = \int_{-\infty}^{\infty} \overbrace{p(x)}^{\text{Constant}}\log \left({\frac {\overbrace{p(x)}^{\text{Constant}}}{\underbrace{q(x)}_{Variable}}}\right)\,dx
  \end{equation*}
  \begin{itemize}
    \item The Reverse KL will sample from $Q$, and evaluate the log probabilities from $P$ and $Q$
    \item Recall: KL Divergence is not symmetric, and this has drastic implications...
  \end{itemize}
\end{frame}


\begin{frame}
  \frametitle{Digression: Differentiable Sampling via the Reparameterization Trick}

\end{frame}


\begin{frame}
  \frametitle{Digression: Common Reparameterization Tricks}

\end{frame}


\begin{frame}
  \frametitle{Reverse KL: Learning a Bimodal}

\end{frame}


\begin{frame}
  \frametitle{Reverse KL: Learning a Bimodal (Attempt 2)}

\end{frame}


\begin{frame}
  \frametitle{Reverse KL: Zero-Forcing}

\end{frame}


\begin{frame}
  \frametitle{Jensen - Shannon Divergence: A Symmetric Divergence}

\end{frame}


\begin{frame}
  \frametitle{Jensen - Shannon Divergence: Training}

\end{frame}


\begin{frame}
  \frametitle{A Family of Divergences: $f$-Divergence}

\end{frame}


\begin{frame}
  \frametitle{$f$-Divergence: Total Variation}

\end{frame}


\begin{frame}
  \frametitle{Other Meterics: Earth Mover's Distance}

\end{frame}


\begin{frame}
  \frametitle{Summary of Methods}
  \begin{center}
  \begin{tabular}{lC{1.5cm}|C{1.5cm}|C{1.5cm}|C{1.5cm}}
  \toprule
  {} & \multicolumn{2}{c}{$P$} & \multicolumn{2}{c}{$Q$} \\
  \cmidrule(lr){2-3}  \cmidrule(lr){4-5}
  {} &  \multicolumn{1}{c}{$\log p(x)$} &  \multicolumn{1}{c|}{$x \sim P$} &  \multicolumn{1}{c}{$\log q(x)$} &  \multicolumn{1}{c}{$x \sim Q$} \\
  \midrule
  Cross-Entropy    &        & \cmark & \cmark &        \\
  Forward KL       & \cmark & \cmark & \cmark &        \\
  Reverse KL       & \cmark &        & \cmark & \cmark \\
  JS Divergence    & \cmark & \cmark & \cmark & \cmark \\
  $f$-Divergence   & \cmark & \cmark & \cmark & \cmark \\
  \bottomrule
  \end{tabular}
  \end{center}
\end{frame}


\begin{frame}
  \frametitle{Potpurri: Advanced Techniques}
  \begin{center}
  \begin{tabular}{lC{1.5cm}|C{1.5cm}|C{1.5cm}|C{1.5cm}}
  \toprule
  {} & \multicolumn{2}{c}{$P$} & \multicolumn{2}{c}{$Q$} \\
  \cmidrule(lr){2-3}  \cmidrule(lr){4-5}
  {} &  \multicolumn{1}{c}{$\log p(x)$} &  \multicolumn{1}{c|}{$x \sim P$} &  \multicolumn{1}{c}{$\log q(x)$} &  \multicolumn{1}{c}{$x \sim Q$} \\
  \midrule
  Cross-Entropy    &        & \cmark & \cmark &        \\
  Forward KL       & \cmark & \cmark & \cmark &        \\
  Reverse KL       & \cmark &        & \cmark & \cmark \\
  JS Divergence    & \cmark & \cmark & \cmark & \cmark \\
  $f$-Divergence   & \cmark & \cmark & \cmark & \cmark \\
  \midrule
  Adversarial      &        & \cmark &        & \cmark \\
  \bottomrule
  \end{tabular}
  \end{center}
\end{frame}


\begin{frame}
  \frametitle{Potpurri: Advanced Techniques}
  \begin{enumerate}
    \item Variational Inference with ELBO
    \item Transforms
    \item Adversarial Training
    \item MCMC Methods
  \end{enumerate}

\end{frame}







\end{document}
