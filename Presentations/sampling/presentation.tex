\documentclass{beamer}
\usepackage[utf8]{inputenc} % allow utf-8 input
\usepackage[T1]{fontenc}    % use 8-bit T1 fonts
\usepackage{hyperref}       % hyperlinks
\usepackage{url}            % simple URL typesetting
\usepackage{booktabs}       % professional-quality tables
\usepackage{amsfonts}       % blackboard math symbols
\usepackage{nicefrac}       % compact symbols for 1/2, etc.
\usepackage{microtype}      % microtypography
\usepackage{amsmath}
\usepackage{tikz}
\usetikzlibrary{calc}
\usetikzlibrary{bayesnet}
\usetikzlibrary{arrows}
\usepackage{color}
\usepackage{array}
\usepackage{dsfont}
\usepackage{multirow, graphicx}
 \usepackage{float}
\newcolumntype{C}[1]{>{\centering\arraybackslash}p{#1}}
\newcolumntype{R}[1]{>{\raggedleft\arraybackslash}p{#1}}
\newcolumntype{L}[1]{>{\raggedright\arraybackslash}p{#1}}
\usepackage{caption}
\usepackage{subfig}
\usepackage{pifont}
\usepackage{xcolor}
\usepackage{algorithm,algorithmic}
% \floatname{algorithm}{Procedure}
\renewcommand{\algorithmicrequire}{\textbf{Input:}}
\renewcommand{\algorithmicensure}{\textbf{Output:}}
\newcommand{\cmark}{\textcolor{green!80!black}{\ding{51}}}
\newcommand{\xmark}{\textcolor{red}{\ding{55}}}
\DeclareMathOperator*{\argmin}{argmin}
\urlstyle{same}
\usepackage{listings}
% \usetheme{Boadilla}

\title{Introduction to Random Numbers, Sampling,\\and MCMC Methods}
% \subtitle{Using Beamer}
\author{Bill Watson}
\institute{S\&P Global}
\date{\today}

\begin{document}

\begin{frame}
\titlepage
\end{frame}


\begin{frame}
\frametitle{What is Sampling and why is it useful?}
\begin{itemize}
  \item Sampling is the practice of generating observations from a population
  \item Monte Carlo methods are algorithms that rely on repeated random sampling
    to obtain approximations where it is difficult or impossible to use deterministic approaches
    \begin{itemize}
      \item Optimization
      \item Numerical Integration
      \item Sampling distributions
    \end{itemize}
\end{itemize}
\end{frame}


\begin{frame}
\frametitle{Application: Approximating $\pi$}
\begin{algorithm}[H]
\begin{algorithmic}[1]
  \REQUIRE Batch Size $N$
  \STATE Sample $u_1 \sim \text{Uniform}(0, 1)$ $N$ times
  \STATE Sample $u_2 \sim \text{Uniform}(0, 1)$ $N$ times
  \STATE $\tilde{\pi} = \frac{4}{N} \cdot \left\vert \left\{ (u_1, u_2) \; \Big \vert \; \sqrt{u_1^2 + u_2^2} < 1 \right\} \right\vert$
  \ENSURE $\tilde{\pi}$
\end{algorithmic}
\caption{Approximating $\pi$}
\end{algorithm}
\end{frame}


\begin{frame}
  \frametitle{Application: Approximating $\pi$}
  \begin{figure}
    \centering
    \includegraphics[width=\textwidth]{assets/approx_pi_circle}
    \caption{The ratio of points inside the unit circle approximates $\frac{\pi}{4}$}
  \end{figure}
\end{frame}


\begin{frame}
  \frametitle{Application: Approximating $\pi$}
  \begin{figure}
    \centering
    \includegraphics[width=\textwidth]{assets/approx_pi_error}
  \end{figure}
\end{frame}


%%%%%%%%%%%%%%%%%%%%%%%%%%%%%%%%%%%%%%%%%%%%%%%%%%%%%%%%%%%%%%%%%%%%%%%%%%%%%%%%


\begin{frame}
\frametitle{What can we do with our samples?}
\begin{itemize}
  \item Integration: $\int p(x) \, f(x) \, dx = \frac{1}{n} \sum_{x_i \sim P} f(x_i) $
  \item Expectation: $ \mu = \frac{1}{n} \sum_{x_i} x_i $
  \item Variance: $ \sigma^2 = \frac{1}{n} \sum_{x_i} \left( x_i - \mu \right)^2$
  \item Median: $\text{median} = \text{median} (x_1, x_2, \hdots x_n)$
  \item Entropy: $\mathbb{H}(P) = - \frac{1}{n} \sum_{x_i \sim P} \log p (x_i)$
  \item CDF: $p(c) = \frac{1}{n} \, \vert \{ x_i \, | \, x_i \leq c \} \vert$
\end{itemize}
\end{frame}



%%%%%%%%%%%%%%%%%%%%%%%%%%%%%%%%%%%%%%%%%%%%%%%%%%%%%%%%%%%%%%%%%%%%%%%%%%%%%%%%


\begin{frame}
\frametitle{Pseudo-Random Number Generators}
\end{frame}


\begin{frame}
\frametitle{Pseudo-Random Number Generators: LCG}
\begin{algorithm}[H]
\begin{algorithmic}[1]
  \REQUIRE Modulus $m$, Multiplier $a$, Increment $c$, Seed $X_0$
  \STATE $X_{i+1} = \left( a \cdot X_i + c \right) \mod m$
  \ENSURE $X_{i+1}$
\end{algorithmic}
\caption{Linear Congruential Generator}
\end{algorithm}
\end{frame}


\begin{frame}
\frametitle{Pseudo-Random Number Generators: Mersenne Twister}
\end{frame}


%%%%%%%%%%%%%%%%%%%%%%%%%%%%%%%%%%%%%%%%%%%%%%%%%%%%%%%%%%%%%%%%%%%%%%%%%%%%%%%%


\begin{frame}
\frametitle{Generating a Normal from the CDF}
\begin{itemize}
  \item We can use the cumulative distribution function to sample any distribution
  \item For instance, a normal's CDF is:
\end{itemize}
\begin{equation*}
  F(x) = \frac{1}{2} \left[ 1 + \text{erf}\left( \frac{x - \mu}{\sigma \sqrt{2}} \right) \right]
\end{equation*}
\begin{itemize}
  \item With an Inverse CDF as:
\end{itemize}
\begin{equation*}
  F^{-1}(p) = \mu + \sigma \sqrt{2} \cdot \text{erf}^{-1}\left( 2p - 1 \right)
\end{equation*}
\begin{itemize}
  \item $\text{erf}(x)$ is the error function, defined as:
\end{itemize}
\begin{equation*}
  \text{erf}(x) = \frac{2}{\sqrt{\pi}} \int^{x}_{0} e^{-t^2} dt
\end{equation*}
\end{frame}


\begin{frame}
\frametitle{Inverse Transform Sampling}
\begin{itemize}
  \item It's easy to generalize this method to any distribution with a closed-form inverse CDF
  \begin{algorithm}[H]
  \begin{algorithmic}[1]
    \REQUIRE Inverse CDF $F^{-1}$
    \STATE Sample $u \sim \text{Uniform}(0, 1)$
    \STATE $X = F^{-1} (u)$
    \ENSURE $X$
  \end{algorithmic}
  \caption{Inverse Transform Sampling}
  \end{algorithm}
\end{itemize}

\end{frame}


\begin{frame}
\frametitle{Inverse Transform Sampling: Example}
\end{frame}


\begin{frame}
\frametitle{Table of Inverse CDFs for Common Distributions}
\begin{center}
\begin{tabular}{C{5cm}C{5cm}}
\toprule
{Distribution} & $F^{-1}(p)$ \\
\midrule
$\mathcal{N}(\mu, \sigma)$   &  $\mu + \sigma \sqrt{2} \cdot \text{erf}^{-1}\left( 2p - 1 \right)$ \\
\\
$\text{Uniform}(a, b)$   &  $a + p \cdot \left( b - a \right)$ \\
\\
$\text{Exp}(\lambda)$   &  $\frac{- \ln \left( 1 - p \right)}{\lambda}$ \\
\\
\bottomrule
\end{tabular}
\end{center}
\end{frame}


%%%%%%%%%%%%%%%%%%%%%%%%%%%%%%%%%%%%%%%%%%%%%%%%%%%%%%%%%%%%%%%%%%%%%%%%%%%%%%%%


\begin{frame}
\frametitle{Rejection Sampling}
\end{frame}


\begin{frame}
\frametitle{Rejection Sampling: Algorithm}
\begin{algorithm}[H]
\begin{algorithmic}[1]
  \REQUIRE Model $P$, Proposal $Q$, $M > 1$
  \STATE Sample $x \sim P$
  \STATE Sample $u \sim \text{Uniform}(0, 1)$
  \IF {$u < \frac{p(x)}{M \cdot q(x)}$}
    \STATE {Accept $x$}
  \ELSE
    \STATE {Reject $x$}
  \ENDIF
  \ENSURE Accepted Samples
\end{algorithmic}
\caption{Rejection Sampling}
\end{algorithm}
\end{frame}

\begin{frame}
  \frametitle{Rejection Sampling: Intuition}
  \begin{figure}
    \centering
    \includegraphics[width=\textwidth]{assets/rejection_sampling_explanation}
  \end{figure}
\end{frame}


\begin{frame}
  \frametitle{Rejection Sampling: Example}
  \begin{figure}
    \centering
    \includegraphics[width=\textwidth]{assets/rs_m_1_3}
    \caption{Rejection Sampling with $M = 1.3$}
  \end{figure}
\end{frame}

\begin{frame}
  \frametitle{Rejection Sampling: Example}
  \begin{figure}
    \centering
    \includegraphics[width=\textwidth]{assets/rs_hist_1_3}
    \caption{$M = 1.3$, $6,688$ Accepted Samples}
  \end{figure}
\end{frame}

\begin{frame}
  \frametitle{Rejection Sampling: Example}
  \begin{figure}
    \centering
    \includegraphics[width=\textwidth]{assets/rs_m_2_5}
    \caption{Rejection Sampling with $M = 2.5$}
  \end{figure}
\end{frame}

\begin{frame}
  \frametitle{Rejection Sampling: Example}
  \begin{figure}
    \centering
    \includegraphics[width=\textwidth]{assets/rs_hist_2_5}
    \caption{$M = 2.5$, $4,085$ Accepted Samples}
  \end{figure}
\end{frame}

\begin{frame}
  \frametitle{Rejection Sampling: Example}
  \begin{figure}
    \centering
    \includegraphics[width=\textwidth]{assets/rs_m_100}
    \caption{Rejection Sampling with $M = 100$}
  \end{figure}
\end{frame}

\begin{frame}
  \frametitle{Rejection Sampling: Example}
  \begin{figure}
    \centering
    \includegraphics[width=\textwidth]{assets/rs_hist_100}
    \caption{$M = 100$, $92$ Accepted Samples}
  \end{figure}
\end{frame}


\begin{frame}
\frametitle{Rejection Sampling: Pros \& Cons}
\end{frame}


%%%%%%%%%%%%%%%%%%%%%%%%%%%%%%%%%%%%%%%%%%%%%%%%%%%%%%%%%%%%%%%%%%%%%%%%%%%%%%%%


\begin{frame}
\frametitle{What is MCMC?}
\end{frame}


\begin{frame}
\frametitle{Metropolis-Hastings}
\end{frame}


\begin{frame}
\frametitle{Metropolis-Hastings: Algorithm}
  \begin{algorithm}[H]
  \algsetup{linenosize=\tiny}
  \scriptsize
  \begin{algorithmic}[1]
    \REQUIRE Model $P$, Proposal $Q$
    \STATE Initialize $x_0$
    \FOR {$s = 0, 1, \hdots$}
      \STATE Sample $x' \sim q(x'|x_s)$
      \STATE Compute acceptance probability
        \begin{flalign*}
          r = \min\left( 1, \; \frac{p(x')}{p(x_s)} \frac{q(x_s|x')}{q(x'|x_s)} \right) &&
        \end{flalign*}
      \STATE Sample $u \sim \text{Uniform}(0, 1)$
      \IF {$u < r$}
        \STATE {Accept $x'$}
        \STATE {Set $x_{s+1} = x'$}
      \ELSE
        \STATE{Reject $x'$}
        \STATE {Set $x_{s+1} = x_s$}
      \ENDIF
    \ENDFOR
    \ENSURE Accepted Samples
  \end{algorithmic}
  \caption{Metropolis-Hastings Algorithm}
  \end{algorithm}
\end{frame}


\begin{frame}
\frametitle{Metropolis-Hastings: Example}
\end{frame}


\begin{frame}
\frametitle{Metropolis-Hastings: Conditions}
\end{frame}


%%%%%%%%%%%%%%%%%%%%%%%%%%%%%%%%%%%%%%%%%%%%%%%%%%%%%%%%%%%%%%%%%%%%%%%%%%%%%%%%


\begin{frame}
\frametitle{Variants of Metropolis-Hastings}
\begin{itemize}
  \item Metropolis Algorithm
  \begin{itemize}
    \item Uses a symmetric proposal distribution $Q$, such that: \\
     $q(x_s|x') = q(x'|x_s)$
  \end{itemize}
  \item Metropolis-Adjusted Langevin Algorithm
  \begin{itemize}
    \item New states are proposed with Langevin dynamics: \\
      $x' = x_s + \tau \nabla \log \pi(x_s) + \sqrt{2 \tau} \xi_k$
    \item Proposal probabilties are normally distributed: \\
      $q(x'| x_s) \sim \mathcal{N}(x_s + \tau \nabla \log \pi(x_s), \, 2 \tau  I_d)$
  \end{itemize}
  \item Hamiltonian Monte-Carlo
  \begin{itemize}
    \item Adds Hamiltonian Dynamics
  \end{itemize}
  \item No-U-Turn Sampler (NUTS)
  \begin{itemize}
    \item
  \end{itemize}
\end{itemize}
\end{frame}


%%%%%%%%%%%%%%%%%%%%%%%%%%%%%%%%%%%%%%%%%%%%%%%%%%%%%%%%%%%%%%%%%%%%%%%%%%%%%%%%


\begin{frame}
\frametitle{Gibbs Sampling}
\end{frame}


%%%%%%%%%%%%%%%%%%%%%%%%%%%%%%%%%%%%%%%%%%%%%%%%%%%%%%%%%%%%%%%%%%%%%%%%%%%%%%%%

\begin{frame}
\frametitle{Further Reading}
  \begin{itemize}
    \item Machine Learning: A Probabilistic Perspective by Kevin Murphy
    \item Monte Carlo Methods in Financial Engineering by Paul Glasserman
    \item Probabilistic Graphical Models: Principles and Techniques by Daphne Koller and Nir Friedman
  \end{itemize}
\end{frame}



% Refs, ideas, etc


\end{document}
